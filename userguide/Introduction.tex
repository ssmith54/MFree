\chapter{Introduction}
\section{MFree}
The Mfree libary is a framework developed in go-lang for mesh-free modelling in a research environment. It was developing during my PhD for application to stretch blow moulding a manufacturing technique used to produce polymer bottles. The main features of the code are the ability to simulate non-linear solid mechanics problems using an explicit solver. The use of the MFree library is intended to simplify the use of mesh-free methods within a research context, and further provide a learning resource for meshfree methods. 

To facilitate this goal, this manuscript has been created in order to describe the main features of this code. The code has been intended to be designed around the idea of the interaction between objects, similar to the object oriented style of programming, without the complexities of inheritance and polymorphism present in other langauges such as C++ and Java. Hence, it is the aim of this code is to provide a balance between a user friendly black-box (similar to Abaqus) and a tool for researchers within computational mechanics.

The go-lang langauge has been chosen as it provides inbuilt conncurncy along with a readability, and usability often not found in other high performance langauges (C, Fortran, C++).
\section{Overview}


