\chapter{Design}
\section{Domain}

\subsubsection*{Overview}
The domain package is intended to hold together the geometric details of the problem, i.e the nodes, the intergration cells and the degrees of freedom attached to the domain. From this other packages will refrence this domain, such as the meshfree packaage, which builds the meshfree domain, and the integration package. At present there are two ways to build a domain, manually by adding nodes to a domain object, or more simply by providing a planar-straight line graph (PLSG) to the domain constructor, given by:
\begin{tcolorbox}
\begin{lstlisting}
domain := domain.DomainNew(fileName string, options string, dim int, global_coordinate coordinateSystem)
\end{lstlisting}
\end{tcolorbox}

Where $fileName$ is the name of the PLSG, $options$ provides a set of rules the mesh, and voronoi generator, see \url{https://www.cs.cmu.edu/~quake/triangle.html} for a description. The variable $dim$ ensures that correct number of degrees of freedom(DOFs) will be set. The coordinate system can be generated using the following function (for Cartesian coordinates),
\begin{lstlisting}
globalCS := coordinatesystem.CreateCartesian()
\end{lstlisting}
axisymmetric (cylindrical coordinates) are also supported. This function will construct the nodes, the degrees of freedom (assuming they are all free to start with) and the Voronoi diagram, which is used in the stabalised nodal integration scheme (SCNI).

\subsubsection*{Public functions}

\subsubsection*{Private functions}