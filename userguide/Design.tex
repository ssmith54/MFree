\chapter{Design}
\section{Domain}

\subsubsection*{Overview}
The domain package is intended to hold together the geometric details of the problem, i.e the nodes, the intergration cells and the degrees of freedom attached to the domain. From this other packages will reference this domain, such as the meshfree packaage, which builds the meshfree domain, and the integration package.
The domain is described by the structure 

\begin{tcolorbox}
	\begin{lstlisting}
	type Domain struct {	
		Name string // name the domain
		Nodes []node.Node // nodes within the domain
		num_nodes int // number of nodes
		voronoi *voronoi.Voronoi // Voronoi diagram
		dim int // dimension of domain
		boundaryNodes []int // index of nodes on boundary
		global_basis *[]*Dir //basis vectors e.g <1,0> <0,1> for 2D 
	}
	\end{lstlisting}
\end{tcolorbox}

At present there are two ways to build a domain, manually by adding nodes to a domain object, or more simply by providing a planar-straight line graph (PLSG) to the domain constructor, given by:
\begin{tcolorbox}
\begin{lstlisting}
domain := domain.DomainNew(fileName string, options string, dim int, global_coordinate coordinateSystem)
\end{lstlisting}
\end{tcolorbox}

Where $fileName$ is the name of the PLSG, $options$ provides a set of rules the mesh, and voronoi generator, see \url{https://www.cs.cmu.edu/~quake/triangle.html} for a description. The variable $dim$ ensures that correct number of degrees of freedom(DOFs) will be set. The coordinate system can be generated using the following function (for Cartesian coordinates),
\begin{lstlisting}
globalCS := coordinatesystem.CreateCartesian()
\end{lstlisting}
axisymmetric (cylindrical coordinates) are also supported. This function will construct the nodes, the degrees of freedom (assuming they are all free to start with) and the Voronoi diagram, which is used in the stabalised nodal integration scheme (SCNI).

\subsubsection*{Public functions}
In this section the public functions available to a domain object are described
\begin{lstlisting}
func (domain *Domain) GetDim() int 
\end{lstlisting}
Returns the dimensions of the Domain object $domain$
\begin{lstlisting}
func (domain *Domain) SetCoordinateSystem() 
\end{lstlisting}
Sets the coordinate system of the Domain object $domain$
\begin{lstlisting}
func (domain *Domain) AddNodes(nodes ...*node.Node) 
\end{lstlisting}
Appends the nodes to the domain and increments the number of nodes counter
\begin{lstlisting}
GetNumNodes
\end{lstlisting}
\begin{lstlisting}
GetNodesIn
\end{lstlisting}
\begin{lstlisting}
TriGen
\end{lstlisting}
\begin{lstlisting}
UpdateDomain
\end{lstlisting}
\begin{lstlisting}
CopyDomain
\end{lstlisting}
\begin{lstlisting}
CreatDofs
\end{lstlisting}
\begin{lstlisting}
GenerateClippedVoronoi
\end{lstlisting}

\begin{lstlisting}
GetVoronoi
\end{lstlisting}

\begin{lstlisting}
PrintNodesToImg
\end{lstlisting}



\subsubsection*{Private functions}
None so far